\chapter{Обобщённая постановка задачи}
\label{cha:chap31}

Задача состоит в том, чтобы разработать алгоритм управления скоростью бесколлекторными бездатчиковыми двигателями постоянного тока для достижения заданных в техническом задании показателей качества: максимальное перерегулирование --- 30 \%, максимальное время переходного процесса --- 0,2 с, максимальная установившаяся ошибка --- 1 \%.	

Важными аспектами алгоритма являются возможность работы в условиях неточно (с какой-то погрешностью) известных параметров двигателя (а именно, сопротивление и индуктивность статора), т. е. обладать робастностью по отношению к этим параметрам; иметь низкие затраты в плане вычислительной мощности для реализации на низкопроизводительном оборудовании (например, на микроконтроллере); использовать только значения токов и напряжений фаз двигателя для определения параметров регулирования. Также желательным критерием является простота и гибкость настройки при работе с различными двигателями.

