\chapter{Разрабатываемое техническое решение}
\label{cha:chap3}

\section{Определение принципа работы двигателя и выбор алгоритма управления}

Для исследования был выбран двигатель без установленных датчиков Холла, поэтому необходимо использовать один из приёмов определения положения ротора, рассмотренных в \ref{sec:sensorless_ways}. Был выбран метод на основе наблюдателя противо-ЭДС, из которого можно получить положение ротора и скорость.

Что касается выбора алгоритма управления скоростью, то предпочтение отдаётся методу на основе прямого управления моментом в силу его простоты, робастности и меньшей вычислительной нагрузки по сравнению с другими рассмотренными методами.

Из-за того, что выбранный алгоритм не так ресурсозатратен, нежели другой метод на основе прямого управления моментом --- векторное управление, в условиях ограниченных вычислительных ресурсов можно позволить увеличить частоту измерений и расчётов, что также положительно скажется на динамических свойствах и позволит увеличить точность регулирования.

\section{Необходимые узлы для реализации технического решения}
\label{sec:nodes}

На основе выбранных принципов функционирования устройства и алгоритма управления система должна включать в себя следующие узлы:
\begin{enumerate}
	\item блок с логикой управления;
	\item инвертор, служащий для преобразования ШИМ сигнала и сигнала с системы регулирования в сигналы для управления фазами двигателя;
	\item механизм определения тока и напряжения фаз для функционирования алгоритма;
	\item блок питания двигателя;
	\item БДПТ без установленных датчиков положения.
\end{enumerate}
