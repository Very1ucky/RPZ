\TechTask
\thispagestyle{empty}
Объектом исследования является бесколлекторный двигатель постоянного тока без установленных датчиков Холла, имеющий трапезоидальный характер противо-ЭДС.

Основной целью работы является разработка алгоритма для управления такого рода двигателями.

Для достижение цели поставлены следующие задачи: 
 1. Исследование существующих подходов к управлению и обзор существующих технических решений
 2. Синтез алгоритма и создание модели его реализации в программном комплексе Matlab/Simulink. Требования, предъявялемые к алгоритму: перерегулирование - 30 \%; время переходного процесса - 0,4 с; максимальная установившаяся ошибка - 1 \%
 3. Разработка печатной платы для реализации готового алгоритма в приложении Altium Designer, подготовка файлов к её изготовлению и изготовление. Требование, предъявляемые к разрабатваемой схеме: напряжение питания - 12 В; максимальный ток силовой части - 5 А; максимальная частота ШИМ - 20 кГц; наличие на плате контактов для подключения выходов микроконтроллера, на котором будет реализован алгоритм
 4. Экспериментальная 

Основными задачами работы являются синтез алгоритма управления такими двигателями, его проверка в ходе моделирования и реализация готового устройства, реализуещего полученный алгоритм.

Решение первых двух задач (синтез алгоритма, моделирование) будет проводиться с использованием программного комплекса Matlab/Simulink.

Требования к разрабатываемому алгоритму:
\begin{enumerate}
	\item Перерегулирование - $30$ \%
	\item Время переходного процесса - $0,4$ c
	\item Максимальная установившаяся ошибка - $1$ \% 
\end{enumerate}

Для решения последней задачи будет задействовано приложение Altium Designer для разработки схемы, её проверки и дальнейшей подготовки к изготовлению печатной платы.

Требования к разрабатываемой схеме:
\begin{enumerate}
	\item Напряжение питания схемы и двигателя  - $12$ В, максимальный ток силовой части - $5$ А
	\item Максимальная частота ШИМ - $20$ кГц
	\item Алгоритм управления реализуется на микроконтроллере, поэтому на плате должны быть контакты для подключения его пинов
\end{enumerate}

Также в ходе работы будут дополнительно рассмотрены следующие вопросы:
\begin{enumerate}
	\item Обзор существующих алгоритмов и аналогов разрабатываемого устройства
	\item Специфика работы бесколлекторных двигателей и принцип их управления
\end{enumerate}


