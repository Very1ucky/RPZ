\Introduction
\thispagestyle{empty}

Бесколлекторные двигатели постоянного тока – это тип двигателей постоянного тока, которые обладают большей эффективностью, надежностью и меньшими габаритами по сравнению с традиционными двигателя постоянного тока \cite{book.itmo_motors}. В связи с чем сейчас они обретают всё большую популярность, когда речь идёт о компактности, энергоэффективности и уменьшении веса и используются в большинстве современных электронных устройств, особенно в робототехнике, медицинских приборах и инструментах. 

Однако они обладают и одним недостатком: необходимостью специального драйвера для коммутации обмоток для обеспечения его вращения. В связи с этим усложняются конструкция и эксплуатация. Но с развитием полупроводниковых компонентов эта проблема уходит на второй план и перекрывается преимуществами таких двигателей, которые были описаны ранее.

Поэтому главной задачей и целью данной работы является разработка устройства управления для обеспечения заданных показателей качества. В ходе работы будет синтезирован алгоритм коммутации обмоток двигателя и управления им в программной среде Matlab/Simulink. После чего будет разработано и изготовлено готовое устройство для проверки полученного алгоритма на практике и сравнения с результатами, полученными при моделировании. 

Далее будут рассмотрены существующие алгоритмы управления бесколлекторными двигателями и технические реализации алгоритмов фкнкционирования подобных устройств с целью формирования собственного технического решения.




