\Introduction
\thispagestyle{empty}

Бесколлекторные двигатели постоянного тока – это тип двигателей постоянного тока, которые обладают большей эффективностью, надежностью и меньшими габаритами по сравнению с традиционными двигателя постоянного тока \cite{book.itmo_motors}. В связи с этим они сейчас обретают всё большую популярность, когда речь идёт о компактности, энергоэффективности и уменьшении веса и используются в большинстве современных электронных устройств, применяются во многих сферах, в том числе в робототехнике, медицинских приборах и инструментах. 

Однако стоит отметить и один недостаток: необходимость специального драйвера для обеспечения его вращения и регулирования. В связи с чем усложняются конструкция и эксплуатация (появляется больше необходимых для обслуживания узлов). Но с развитием полупроводниковых компонентов эта проблема уходит на второй план и компенсируется преимуществами таких двигателей, которые были описаны ранее.

Поэтому главной задачей и целью данной работы является разработка системы управления такого рода двигателями для обеспечения заданных показателей качества. В ходе работы будет синтезирован алгоритм коммутации обмоток двигателя и управления им в программной среде Matlab/Simulink. После чего будет разработано и изготовлено готовое устройство для проверки полученного алгоритма на практике. 

Далее будут рассмотрены существующие алгоритмы управления бесколлекторными двигателями и технические реализации алгоритмов функционирования подобных устройств с целью формирования собственного технического решения.




